%\documentclass[a4paper,10pt]{article}
\documentclass{SBCbookchapter}
\usepackage[utf8x]{inputenc}
\usepackage[brazil]{babel}
\usepackage[T1]{fontenc}
\usepackage{setspace}
\usepackage[pdftex]{hyperref}



%opening
\title{MACC - Universidade Estadual do Ceará\\\center{Redes de Computadores - Web Proxy}}
\author{Alisson Barbosa de Souza\\Antônio Sérgio de Sousa Vieira}

\begin{document}
\maketitle

\hyphenation{pro-gra-ma-ção}

\onehalfspace

\section{Introdução}
Um Proxy Web é um programa que atua como um intermediador entre um navegador Web e um servidor Web. Em vez de contactar um servidor Web diretamente para carregar uma página Web, o navegador se conecta a um proxy, que encaminha a requisição ao servidor Web. Quando o servidor Web responde ao proxy, este envia a resposta ao navegador.

O proxy tem várias utilidades. Algumas vezes eles são utilizados em firewalls, já que ele é apenas uma forma do navegador passar pelo firewall para se conectar a um servidor do outro lado do firewall. O proxy também pode converter páginas, fazendo com que elas seja visíveis em telefones celulares, por exemplo. Eles também são utilizados para deixar a nevegação anônima ao retirar toda a informação de identificação da requisição do navegador. O proxy também pode ser utilizado para criar um \textit{cache} de objetos Web.

Neste trabalho foi desenvolvido um proxy simples responsável por encaminhar requisições de clientes a servidores web e encaminhar as respostas dos servidores aos navegadores.

\section{Funcionamento do Proxy}
Basicamente, o funcionamento do proxy se dá inicialmente abrindo um socket e esperando por uma conexão de um cliente qualquer. Quando ele recebe a requisição do cliente, aceita a conexão, ler a requisição HTTP e extrai o nome do servidor. Ele deve então abrir uma conexão ao servidor final, enviar uma requisição, receber a resposta e encaminhá-la ao navegador, se a requisição não for bloqueada.

\section{Requisições HTTP}
Uma requisição HTTP pode ser de muitos tipos, porém, neste trabalho, a única requisição considerada foi a GET. Quando uma URL {\tt http://www.google.com/index.html} é digitada no navegador é aberta uma conexão TCP em {\tt www.google.com} na porta 80 e é enviada uma requisição:

\begin{verbatim}
 GET /index.html HTTP/1.1\r\n
\end{verbatim}

A sequência $\setminus r\setminus n$ representa o caractere ASCII 13 seguido do caractere ASCII 10. Em resposta, o servidor web localizará o arquivo chamado {\tt index.html} e transmitirá do conteúdo da página para o navegador.

Um navegador Web pode ser configurado para utilizador um proxy web em vez de mandar uma requisição direta ao servidor Web. Suponha que um navegador esteja configurado para se comunicar com o proxy. Neste caso, quando quando tenta-se carregar a página {\tt http://www.google.com/index.html}, o navegar envia a seguinte mensagem

\begin{verbatim}
 GET http://www.google.com/index.html HTTP/1.1\r\n
 Host: www.google.com
\end{verbatim}

para o proxy, onde ele encarrega-se de extrair o host e encaminhar a mensagem recebida para servidor Web. Após estabelecida a conexão entre o proxy e o servidor Web, o proxy encaminha a mensagem do navegar e aguarda a resposta do servidor, a medida que o proxy recebe estas mensagens ele as encaminha para o navegador.

\section{Questões de implementação}

\end{document}
